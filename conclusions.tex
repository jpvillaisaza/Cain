\chapter{Conclusions}
\label{chap:conclusions}

\begin{epigraphs}
\qitem{
  ``What dreadful nonsense we \emph{are} talking!''
}{---\textcite[255]{carroll-2004}}
\qitem{
  ``You may call it `nonsense' if you like, but \emph{I've} heard
  nonsense, compared with which that would be as sensible as a
  dictionary!''
}{---\textcite[173]{carroll-2004}}
\end{epigraphs}

\todo{(Future work) The reasoning of Remark
  \ref{re:triple-endofunctor} also applies for an
  \texthaskell{Applicative} type constraint in the \texthaskell{Monad}
  type class. In this case, it makes real practical sense to include
  it.}

%% From Natural transformations

\todo{The relation between natural transformations and parametrically
  polymorphic functions is not as straightforward as discussed here.
  Polymorphic functions actually correspond to lax natural
  transformations, but this fact is beyond the scope of this
  dissertation.}

%% From Algebras

\todo{We are only concerned with inductive types, not coinductive.}




%% 1. Hacia atrás: categorías incompleto sin functores, functores
%% incompleto sin transformaciones naturales.

%% 2. Fin de los tres conceptos básicos de teoría de categorías. Con esto
%% queda abierto el mundo categórico.

%% 3. Hacia adelante: todo requiere transformaciones. Monoidals:
%% functores con trasnformaciones. Mónadas: endofunctores con
%% transformaciones.

%% 4. En transformaciones naturales: abre a parametricidad, visión
%% distinta de polimorfismo.

\clearemptydoublepage
