\chapter{Conclusions}
\label{chap:conclusions}

\begin{epigraphs}
\qitem{
  ``What dreadful nonsense we \emph{are} talking!''
}{---\textcite[255]{carroll-2004}}
\qitem{
  ``You may call it `nonsense' if you like, but \emph{I've} heard
  nonsense, compared with which that would be as sensible as a
  dictionary!''
}{---\textcite[173]{carroll-2004}}
\end{epigraphs}

Our main objective with this project was to study some of the
applications of category theory to functional programming,
particularly in Haskell and Agda, and, more specifically, to describe
and explain the concepts of category theory needed for conceptualizing
and better understanding functors, polymorphism, monads, and algebraic
data types, which we did in Chapters \ref{chap:functors},
\ref{chap:naturals}, \ref{chap:monads}, and \ref{chap:algebras},
respectively. In Chapter \ref{chap:categories}, we identified
categories as the starting point for relating category theory to
functional programming. In the case of algebraic data types and, more
usefully, folds, we identified algebras and initial algebras over
endofunctors as the required concepts for satisfying our main goal; in
the case of functors, the notions of functor and endofunctor; in the
case of monads, the concepts of monad and Kleisli triple; and, in the
case of polymorphism or, more precisely, parametric polymorphism,
natural transformations.

Obviously, we did not cover all of category theory. For instance, we
did not deal with concepts such as adjoints, epimorphisms, limits,
monomorphisms, and universal constructions, which were listed in the
project proposal, but did not answer our purpose. Having said that, we
did cover the trinity of concepts category, functor, and natural
transformation, which is the foundation of all category theory
\parencite[vii]{maclane-1998}, and which creates an opportunity for a
deeper understanding of the subject.

Additionally, some of the applications of category theory to
functional programming are not as straightforward as suggested here.
For example, polymorphic functions actually correspond to lax natural
transformations \parencite[350]{wadler-1989}, and algebraic data types
in Haskell correspond to initial algebras and terminal coalgebras over
endofunctors \parencite[§ 2]{vene-2000}, but such concepts go beyond
the scope of this project. However, our use of category theory seems
to be appropriate and useful, especially from the standpoint of
functional programming.

Although subjective, we believe that this project provides some
interesting examples of how to take advantage of category theory in
functional programming and programming in general, as well as a way to
become a better programmer.

Needless to say, the ideas of category theory might be difficult to
understand at first. As a matter of fact,
\textcite[25]{bird-demoor-1997} claim that ``one does not so much
learn category theory as absorb it over a period of time.'' We claim
that it is definitely worth it.

\section{Future Work}

All unanswered questions and concepts beyond the scope of this project
could be considered as suggestions for future work. For instance, the
questions of Haskell's and Agda's categories, the existence of initial
algebras over endofunctors, and others. We describe some ideas which
we find interesting and appropriate.

\paragraph{Agda}

Whenever suitable for our objective, we used category theory in Agda,
but focused on Haskell. Even though helpful, additional projects could
benefit much more from really exploting a programming language such as
Agda. For instance, we could explore natural transformations and
algebras, or the theorems stating the equivalence between monads and
Kleisli triples.

\paragraph{Applicative functors}

Based on \parencite{mcbride-paterson-2008}, we identified and studied
monoidal functors in order to be able to understand applicative
functors from a category-theoretical point of view, but we did not
include our results here. This seems to be a very relevant next step,
particularly in the context of the ``current, and very likely to
succeed,'' Haskell 2014 \texthaskell{Applicative => Monad}
proposal\footnote{\url{http://www.haskell.org/haskellwiki/Functor-Applicative-Monad_Proposal}.},
which adds an \texthaskell{Applicative} constraint to the
\texthaskell{Monad} type class and promotes \texthaskell{join} to
\texthaskell{Monad}, which we briefly discussed in Remark
\ref{re:monad-bind}.

\paragraph{Beyond catamorphisms and folds}

Based on \parencites{meijer-fokkinga-paterson-1991}{vene-2000}, we
could go beyond catamorphisms and study recursion patterns such as
anamorphisms, hylomorphisms, and paramorphisms. Moreover, we discussed
the \texthaskell{foldr} function, but not \texthaskell{foldl} and
folds in general, which could be a different approach, as well as
those mentioned in \parencite[§ 6]{hutton-1999}.

\paragraph{Category theory}

\textcite{awodey-2010} was a late addition to our references for this
project. The approach and examples of the book seem like a suitable
next step for a more thorough study of category theory and its
applications.

\paragraph{More monads}

As examples of monads, we discussed the \texthaskell{Identity},
\texthaskell{Maybe}, and \texthaskell{[]} monads. However, a more
complete study of monads could include the \texthaskell{Cons},
\texthaskell{IO}, \texthaskell{Reader}, \texthaskell{State}, and
\texthaskell{Writer} monads. Additionally, we could explore the idea
of notions of computation and semantics with \parencite{moggi-1991}.

\paragraph{Parametricity}

In Chapter \ref{chap:naturals}, we discussed quite simple polymorphic
functions, but we could use some of the ideas of
\parencite{wadler-1989} for a deeper understanding of parametricity or
theorems for free. Additionally, \parencite{bernardy-et-al-2012}
provides ideas for analyzing parametricity in Agda.

\paragraph{Parser combinators}

Based on, for instance, \parencite[§ 16]{osullivan-et-al-2008}, we
could study the applications of category theory to parser combinators.

\paragraph{Real categories}

In Sections \ref{sec:category-haskell} and \ref{sec:category-agda}, we
stated that \hask and \agda are not attempts to answer the questions
of Haskell's and Agda's categories, respectively. In the case of
Haskell, we could begin with the study of the category of complete
partial orders \parencite[12.43]{hudak-et-al-2007} and more details
about the problem with bottom values.

\paragraph{The typeclassopedia}

We explored only two of the standard Haskell type classes related to
category theory: \texthaskell{Functor} and \texthaskell{Monad}. Based
on \parencite{yorgey-2009}, we could do the same with
\texthaskell{Arrow}, \texthaskell{Applicative} (see above),
\texthaskell{Category}, \texthaskell{Foldable}, \texthaskell{Monoid},
\texthaskell{Traversable}, among other classes.

\clearemptydoublepage
