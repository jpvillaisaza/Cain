\chapter{Conclusions}
\label{chap:conclusions}

\begin{epigraphs}
\qitem{
  ``What dreadful nonsense we \emph{are} talking!''
}{---\textcite[255]{carroll-2004}}
\qitem{
  ``You may call it `nonsense' if you like, but \emph{I've} heard
  nonsense, compared with which that would be as sensible as a
  dictionary!''
}{---\textcite[173]{carroll-2004}}
\end{epigraphs}

Conclusions.

%% 1. Hacia atrás: categorías incompleto sin functores, functores
%% incompleto sin transformaciones naturales.

%% 2. Fin de los tres conceptos básicos de teoría de categorías. Con esto
%% queda abierto el mundo categórico.

%% 3. Hacia adelante: todo requiere transformaciones. Monoidals:
%% functores con trasnformaciones. Mónadas: endofunctores con
%% transformaciones.

%% 4. En transformaciones naturales: abre a parametricidad, visión
%% distinta de polimorfismo.

\section{Future Work}

TODO: The reasoning of Remark I deleted, which says that it is more
theoretically correct to not include a Functor constraint in the type
class declaration of monads, also applies for an
\texthaskell{Applicative} type constraint in the \texthaskell{Monad}
type class. In this case, it makes real practical sense to include
it.

Applicative functors: \parencite{mcbride-paterson-2008}

TODO: Regarding natural transformations, the relation between natural
transformations and parametrically polymorphic functions is not as
straightforward as discussed here. Polymorphic functions actually
correspond to lax natural transformations, but this fact is beyond the
scope of this dissertation.

TODO: Regarding Algebras, we are only concerned with inductive types,
not coinductive.

\paragraph{Scope}

This study of category theory is restricted to the following concepts:
This list is based on Mac Lane, Pierce, and Poigné.

\begin{enumerate}
\item Introductory concepts and infrastructure of category theory
  \begin{enumerate}
  \item Categories
  \item Infrastructure of category theory
    \begin{itemize}
    \item Monomorphisms, epimorphisms, and isomorphisms
    \item Initial and terminal objects
    \item Universal constructions
    \end{itemize}
  \end{enumerate}
\item Basic concepts of category theory
  \begin{enumerate}
  \item Functors
  \item Natural transformations
  \item Adjoints
  \item Limits
  \end{enumerate}
\item More specific concepts of category theory
  \begin{enumerate}
  \item Algebras
  \item Monads
  \end{enumerate}
\end{enumerate}

\clearemptydoublepage
