\chapter{Conclusions}
\label{chap:conclusions}

\begin{epigraphs}
\qitem{
  ``What dreadful nonsense we \emph{are} talking!''
}{---\textcite[255]{carroll-2004}}
\qitem{
  ``You may call it `nonsense' if you like, but \emph{I've} heard
  nonsense, compared with which that would be as sensible as a
  dictionary!''
}{---\textcite[173]{carroll-2004}}
\end{epigraphs}

Our main objective with this project was to study some of the
applications of category theory to functional programming in Haskell
and Agda, and, more specifically, to describe and explain the concepts
of category theory needed for conceptualizing and better understanding
algebraic data types, functors, monads, and polymorphism, which we did
in Chapters \ref{chap:algebras}, \ref{chap:functors},
\ref{chap:monads}, and \ref{chap:naturals}, respectively. In Chapter
\ref{chap:categories}, we identified categories as the starting point
for relating category theory to functional programming. In the case of
algebraic data types and, more usefully, folds, we identified algebras
and initial algebras over endofunctors as the required concepts for
satisfying our main goal; in the case of functors, the notions of
functor and endofunctor; in the case of monads, the concepts of monad
and Kleisli triple; and, in the case of polymorphism or, more
precisely, parametric polymorphism, natural transformations.

Obviously, we did not cover all of category theory. For instance, we
did not deal with concepts such as adjoints, epimorphisms, limits,
monomorphisms, and universal constructions, which were listed in the
project proposal, but did not answer our purpose. Having said that, we
did cover the trinity of concepts category, functor, and natural
transformation, which is the foundation of all category theory, and
which creates an opportunity for a deeper understanding of the
subject.

Additionally, some of the applications of category theory to
functional programming are not as straightforward as suggested here.
For example, polymorphic functions actually correspond to lax natural
transformations, and algebraic data types in Haskell correspond to
initial algebras and terminal coalgebras over endofunctors, but such
concepts go beyond the scope of this project. However, our use of
category theory seems to be appropriate and useful from the standpoint
of functional programming.

In this regard, we stated that one can be a perfectly competent
functional programmer without knowledge of category theory, and that
categorical concepts can be applied to functional programming with the
intention of, for instance, becoming a better programmer. Although
subjective, we believe that this project provides some interesting
examples of how to take advantage of category theory in functional
programming and programming in general.

Needless to say, the ideas of category theory might be difficult to
understand at first. As a matter of fact,
\textcite[25]{bird-demoor-1997} claim that ``one does not so much
learn category theory as absorb it over a period of time.'' We claim
that it is definitely worth it.

\section{Future Work}

\paragraph{Agda}

Queda una pregunta de si lo que se hizo con Agda es relevante o solo
interesante. We applied all of these to Agda whenever appropriate.

\paragraph{Applicative functors}

Una primera cosa son los functores aplicativos, que estudiamos un
tiempo con functores monoidales de teoría de categorías a partir del
paper de McBride y Paterson. Nuestra discusión sobre mónadas abre las
puertas a algo más avanzado e interesante sobre la relación entre las
clases functor, mónada y aplicativo. TODO: The reasoning of Remark I
deleted, which says that it is more theoretically correct to not
include a Functor constraint in the type class declaration of monads,
also applies for an \texthaskell{Applicative} type constraint in the
\texthaskell{Monad} type class. In this case, it makes real practical
sense to include it. Applicative functors:
\parencite{mcbride-paterson-2008}. This article introduces applicative
functors and identifies the categorical structure of applicative
functors and examines their relationship with monads.

\paragraph{Existence of initial algebras}

From Vene: The initial F-algebra may or may not exist. It is
guaranteed to exist if F is ω-cocontinuous (that is, it preserves the
colimits of ω-chains). All polynomial functors (that is, functors
built from products, sums, the identity functor, and constant
functors) are ω-cocontinuous and, hence, the initial algebras for them
exist. From Andrés: A primera vista parece que todos los functores de
Haskell satisfacen esta condición. Sin embargo, podemos dejar para la
sección de Future Work un comentario al respecto.

\clearemptydoublepage
