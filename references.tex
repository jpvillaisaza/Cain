%% From main.tex:
%% \appendix

%% \chapter{References}
\label{chap:references}

\todo{\parencite{marquis-2013}, which includes: general definitions,
  examples, and applications; brief historical sketch; philosophical
  significance; and a programmatic reading guide.}

\todo{Regarding natural transformations (Chapter \ref{chap:naturals}),
  \parencite{eilenberg-maclane-1945} defines them on two arguments (for
  concordant functors covariant in one argument and contravariant in
  the other). If the second argument (that is, the contravariant one)
  is ignored, the definition is very similar to that of
  \parencite{maclane-1998}, which is the basis of Definition
  \ref{def:natural}. This gives a good idea of the evolution of the
  definition of ``natural transformation.''}

\todo{nLab and MathWorld.}

\todo{Haskell's wiki.}

\todo{Oxford Dictionaries and Merriam-Webster.}


%% TODO citation for the frontispiece


\paragraph{Frontispiece}

The illustration in page \pageref{fig:hatter}, John Tenniel's
\emph{Hatter engaging in rhetoric}, is taken from the Tenniel
illustrations for Lewis Carroll's \emph{Alice's Adventures in
  Wonderland}\footnote{\url{http://www.gutenberg.org/ebooks/114}.}.

\nocite{*}


%% Mac Lane (1998) es la referencia estándar de teoría de categorías.
%% Aunque el texto no ha sido estudiado directamente, todos los textos
%% relacionados con teoría de categorías lo señalan como la referencia
%% estándar y como la guía principal para estudiar teoría de categorías
%% desde las matemáticas.

%% Poigné (1992) es una referencia para estudiar las aplicaciones de
%% teoría de categorías a informática. Fue el texto guía consultado para
%% la elaboración de esta propuesta y está basado en la notación de la
%% primera edición de Mac Lane (1998). El estudio preliminar permite
%% concluir que es una buena referencia central para el desarrollo del
%% proyecto.

%% Pierce (1991) es una referencia introductoria para estudiar las
%% aplicaciones de teoría de categorías a informática. Es una resumen de
%% teoría de categorías y de algunas de sus aplicaciones, y contiene una
%% bibliografía anotada para encontrar más referencias sobre el tema.

%% Bird y de Moor (1997) es una referencia estándar para estudiar las
%% aplicaciones de la teoría de categorías a la informática. Algunas
%% partes de dicho libro fueron consultadas para la elaboración de esta
%% propuesta. Como es el texto guía de algunos cursos de teoría de
%% categorías, también es una de las referencias centrales para el
%% desarrollo del proyecto.

%% Marquis (2010) es un artículo introductorio y filosófico de teoría de
%% categorías en general. Es un buen artículo para el estudio preliminar
%% del tema y contiene una extensa bibliografía anotada para encontrar
%% las referencias más importnantes en los distintos enfoques de teoría
%% de categorías.

\clearemptydoublepage


\chapter{References}
\label{chap:references}

\todo{\parencite{marquis-2013}, which includes: general definitions,
  examples, and applications; brief historical sketch; philosophical
  significance; and a programmatic reading guide.}

\todo{Regarding natural transformations (Chapter \ref{chap:naturals}),
  \parencite{eilenberg-maclane-1945} defines them on two arguments (for
  concordant functors covariant in one argument and contravariant in
  the other). If the second argument (that is, the contravariant one)
  is ignored, the definition is very similar to that of
  \parencite{maclane-1998}, which is the basis of Definition
  \ref{def:natural}. This gives a good idea of the evolution of the
  definition of ``natural transformation.''}

\todo{nLab and MathWorld.}

\todo{Haskell's wiki.}

\todo{Oxford Dictionaries and Merriam-Webster.}


%% TODO citation for the frontispiece


\paragraph{Frontispiece}

The illustration in page \pageref{fig:hatter}, John Tenniel's
\emph{Hatter engaging in rhetoric}, is taken from the Tenniel
illustrations for Lewis Carroll's \emph{Alice's Adventures in
  Wonderland}\footnote{\url{http://www.gutenberg.org/ebooks/114}.}.

\nocite{*}


%% Mac Lane (1998) es la referencia estándar de teoría de categorías.
%% Aunque el texto no ha sido estudiado directamente, todos los textos
%% relacionados con teoría de categorías lo señalan como la referencia
%% estándar y como la guía principal para estudiar teoría de categorías
%% desde las matemáticas.

%% Poigné (1992) es una referencia para estudiar las aplicaciones de
%% teoría de categorías a informática. Fue el texto guía consultado para
%% la elaboración de esta propuesta y está basado en la notación de la
%% primera edición de Mac Lane (1998). El estudio preliminar permite
%% concluir que es una buena referencia central para el desarrollo del
%% proyecto.

%% Pierce (1991) es una referencia introductoria para estudiar las
%% aplicaciones de teoría de categorías a informática. Es una resumen de
%% teoría de categorías y de algunas de sus aplicaciones, y contiene una
%% bibliografía anotada para encontrar más referencias sobre el tema.

%% Bird y de Moor (1997) es una referencia estándar para estudiar las
%% aplicaciones de la teoría de categorías a la informática. Algunas
%% partes de dicho libro fueron consultadas para la elaboración de esta
%% propuesta. Como es el texto guía de algunos cursos de teoría de
%% categorías, también es una de las referencias centrales para el
%% desarrollo del proyecto.

%% Marquis (2010) es un artículo introductorio y filosófico de teoría de
%% categorías en general. Es un buen artículo para el estudio preliminar
%% del tema y contiene una extensa bibliografía anotada para encontrar
%% las referencias más importnantes en los distintos enfoques de teoría
%% de categorías.

\clearemptydoublepage
