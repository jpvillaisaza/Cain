\chapter{Functors}
\label{chap:functors}

\epigraph{
  \emph{Category} has been defined in order to be able to
  define \emph{functor}...
}{---\textcite[18]{maclane-1998}}

In this chapter we explore mathematical functors, and functors in
Haskell and Agda. Mapping over lists, which is accomplished with the
\texthaskell{map} function, is ``a dominant idiom in Haskell''
\parencite[146]{lipovaca-2011}. The type signature of the
\texthaskell{map} function is:
\begin{codehaskell}
map :: (a -> b) -> [a] -> [b]
\end{codehaskell}
According to \parencite[190]{marlow-2010}, given a function
\texthaskell{f} and a list \texthaskell{xs}, \texthaskell{map f xs} is
the list obtained by applying \texthaskell{f} to each element in
\texthaskell{xs}. In other words:
\begin{codehaskell}
map f [x1, x2, ..., xn] = [f x1, f x2, ..., f xn]
\end{codehaskell}
Or, better:
\begin{codehaskell}
map f [x1, x2, ...] = [f x1, f x2, ...]
\end{codehaskell}
The definition of the \texthaskell{map} function is:
\begin{codehaskell}
map :: (a -> b) -> [a] -> [b]
map _ []     = []
map f (x:xs) = f x : map f xs
\end{codehaskell}

Even though this is the correct definition of the \texthaskell{map}
function (it applies a function to all the elements of a list), it is
possible to implement alternative definitions. For instance:
\begin{codehaskell}
map :: (a -> b) -> [a] -> [b]
map _ []     = []
map f (x:xs) = f x : f x : map f xs
\end{codehaskell}
This alternative \texthaskell{map} function applies a function to each
element in a list and duplicates each result.

Deciding whether the former or the latter \texthaskell{map} function
is the correct one for mapping over lists requires a more general
approach. This is achieved with the definition of the
\texthaskell{Functor} type class, which is used for types that can be
mapped over and which generalizes the \texthaskell{map} function as a
\emph{uniform} action over a parameterized type such as
\texthaskell{[a]} or \texthaskell{Maybe a}.

However, the definition of the \texthaskell{Functor} type class is not
enough to determine what a ``uniform action over a parameterized
type'' is. On the other hand, a comment in \parencite[p.
  88]{peytonjones-2003} states that all instances of the
\texthaskell{Functor} type class \emph{should} satisfy the functor
laws. These laws, which are not part of the definition of functors in
Haskell, guarantee that a ``uniform action over a parameterized type''
is actually uniform.

Functors in Haskell implement mathematical functors (that is, functors
in category theory) and the functor laws correspond to the conditions
that a mathematical functor \emph{must} satisfy in order to be a
functor. Studying mathematical functors may not be necessary for
uniformly mapping over a parameterized type, but it may be very useful
for better understanding what that means.

\section{Functors}
\label{sec:functors}

Basically, a functor or morphism of categories maps the objects and
morphisms of a category to objects and morphisms of another category.

\begin{definition}
  \label{def:functor}

  %% \parencites[13]{maclane-1998}[428]{poigne-1992}

  Let $\cat{C}$ and $\cat{D}$ be categories. A functor $\func{F}:
  \cat{C} \to \cat{D}$ assigns to each object $a$ in $\cat{C}$ an
  object $\funcO{F}(a)$ in $\cat{D}$, and to each morphism $f: a \to
  b$ in $\cat{C}$ a morphism $\funcM{F}(f): \funcO{F}(a) \to
  \funcO{F}(b)$ in $\cat{D}$, such that, for all objects $a$ in
  $\cat{C}$,
  \begin{equation}
    \label{eq:functor-identity}
    \funcM{F}(\idO{a}) = \idO{\funcO{F}(a)}
    \text{,}
  \end{equation}
  and, for all morphisms $f: a \to b$ and $g: b \to c$ in $\cat{C}$,
  \begin{equation}
    \label{eq:functor-composition}
    \funcM{F}(g \comp f) = \funcM{F}(g) \comp \funcM{F}(f)
    \text{.}
  \end{equation}
  A functor from a category to itself is called an endofunctor.

\end{definition}

A simple example of functors is the power set operation, which yields
an endofunctor in \set.

\begin{example}
  \label{ex:functor-power-set}

  %% \parencites[13]{maclane-1998}[10--11]{marquis-2013}[431]{poigne-1992}

  The power set endofunctor $\func{P}: \set \to \set$ assigns to each
  set $A$ the set of all subsets of $A$, that is,
  \begin{equation}
    \label{eq:functor-power-set-object}
    \funcO{P}(A) = \{X \mid X \subseteq A\}
    \text{,}
  \end{equation}
  and to each function $f: A \to B$ a function $\funcM{P}(f):
  \funcO{P}(A) \to \funcO{P}(B)$ such that, for all $X \in
  \funcO{P}(A)$,
  \begin{equation}
    \label{eq:functor-power-set-morphism}
    \funcM{P}(f)(X) = \{f(x) \mid x \in X\}
    \text{.}
  \end{equation}
  In order to see that this defines a functor, we prove, in the first
  place, that \eqref{eq:functor-identity} holds for all $X \in
  \funcO{P}(A)$:
  \begin{steps}
    \step{$\funcM{P}(\idO{A})(X)$}
      \eqby{\eqref{eq:functor-power-set-morphism} with $f = \idO{A}$}
    \step{$\{\idO{A}(x) \mid x \in X\}$}
      \eqby{\eqref{eq:set-identity}}
    \step{$\{x \mid x \in X\}$}
      \eqbynothing
    \step{$X$}
      \eqby{\eqref{eq:set-identity} with $A = \funcO{P}(A)$ and $x = X$}
    \step{$\idO{\funcO{P}(A)}(X)$}
  \end{steps}
  In the second place, we prove that \eqref{eq:functor-composition}
  holds for all $X \in \funcO{P}(A)$:
  \begin{steps}
    \step{$\funcM{P}(g \comp f)(X)$}
      \eqby{\eqref{eq:functor-power-set-morphism} with $f = g \comp f$}
    \step{$\{(g \comp f)(x) \mid x \in X\}$}
      \eqby{\eqref{eq:set-composition}}
    \step{$\{g(f(x)) \mid x \in X\}$}
      \eqby{\eqref{eq:functor-power-set-morphism} with $f = g$ and $X = \{f(x) \mid x \in X\}$}
    \step{$\funcM{P}(g)(\{f(x) \mid x \in X\})$}
      \eqby{\eqref{eq:functor-power-set-morphism}}
    \step{$\funcM{P}(g)(\funcM{P}(f)(X))$}
      \eqby{\eqref{eq:set-composition} with $f = \funcM{P}(f)$, $g = \funcM{P}(g)$, and $x = X$}
    \step{$(\funcM{P}(g) \comp \funcM{P}(f))(X)$}
  \end{steps}

\end{example}

``There are many functors between two given categories, and the
question of how they are connected suggests itself''
\parencite[11]{marquis-2013}. For instance, there is always the
identity endofunctor from a category to itself.

\begin{example}
  \label{ex:functor-identity}

  %% \parencite[11]{marquis-2013}

  Let $\cat{C}$ be a category. The identity endofunctor $\func{I}:
  \cat{C} \to \cat{C}$ sends all objects and morphisms of \cat{C} to
  themselves. In detail, for all objects $a$,
  \begin{equation}
    \label{eq:functor-identity-object}
    \funcO{I}(a) = a
    \text{,}
  \end{equation}
  and, for all morphisms $f$,
  \begin{equation}
    \label{eq:functor-identity-morphism}
    \funcM{I}(f) = f
    \text{.}
  \end{equation}
  In order to see that this defines a functor, we prove, in the first
  place, that \eqref{eq:functor-identity} holds:
  \begin{steps}
    \step{$\funcM{I}(\idO{a})$}
      \eqby{\eqref{eq:functor-identity-morphism} with $f = \idO{a}$}
    \step{$\idO{a}$}
      \eqby{\eqref{eq:functor-identity-object}}
    \step{$\idO{\funcO{I}(a)}$}
  \end{steps}
  In the second place, we prove that \eqref{eq:functor-composition}
  holds:
  \begin{steps}
    \step{$\funcM{I}(g \comp f)$}
      \eqby{\eqref{eq:functor-identity-morphism} with $f = g \comp f$}
    \step{$g \comp f$}
      \eqby{\eqref{eq:functor-identity-morphism} with $f = g$ and \eqref{eq:functor-identity-morphism}}
    \step{$\funcM{I}(g) \comp \funcM{I}(f)$}
  \end{steps}

\end{example}

\section{Functors in Haskell}
\label{sec:functors-haskell}

Functors in Haskell are defined by ``the most basic and ubiquitous
type class in the Haskell libraries'' \parencite[18]{yorgey-2009}, the
\texthaskell{Functor} type class, which is exported by the Haskell
Prelude:
\begin{codehaskell}
class Functor f where
  fmap :: (a -> b) -> f a -> f b
\end{codehaskell}
It is used for types that can be mapped over, and generalizes the
\texthaskell{map} function on lists with a uniform action over a
parameterized type. Intuitively, functors represent containers or
computational contexts, but we are more interested in their definition
and its relation to that of mathematical functors.

It is important to note that \texthaskell{f} is a type constructor
rather than a type (its kind is \texthaskell{* -> *}, not
\texthaskell{*}). \texthaskell{[]} and \texthaskell{Maybe} are
examples of such type constructors. The result of applying a type
constructor to a type (for example, \texthaskell{Int} or
\texthaskell{Bool}) is a type or concrete type (that is, something of
kind \texthaskell{*}). Therefore, the kinds of \texthaskell{[] Int}
(that is, \texthaskell{[Int]}) and \texthaskell{Maybe Int} are both
\texthaskell{*}. As another example, since the kind of
\texthaskell{Either} is \texthaskell{* -> * -> *}, it cannot be
declared as an instance of the \texthaskell{Functor} type class.
However, a type constructor can be partially applied and something
like \texthaskell{Either a} can be declared as an instance of
\texthaskell{Functor}.

The fact that \texthaskell{f} is a type constructor can be made
explicit using the \texthaskell{KindSignatures} language option:
\begin{codehaskell}
class Functor (f :: * -> *) where
  fmap :: (a -> b) -> f a -> f b
\end{codehaskell}
The kind signature of \texthaskell{f} shows that it corresponds to the
object mapping of a mathematical functor (that is, \funcO{F}): it
sends objects of \hask (types) to objects of \hask (types).

The \texthaskell{fmap} function is curried and can be rewritten with
extra (and unnecessary) parentheses:
\begin{codehaskell}
class Functor (f :: * -> *) where
  fmap :: (a -> b) -> (f a -> f b)
\end{codehaskell}
The type of \texthaskell{fmap} shows that it corresponds to the
morphism mapping of a mathematical functor (that is, \funcM{F}): it
sends morphisms of \hask (functions) to morphisms of \hask
(functions).

Instances of the \texthaskell{Functor} type class correspond to
functors (more precisely, endofunctors) from \hask to \hask with the
type constructor and the \texthaskell{fmap} function as the required
object and morphism mappings.

Although functors \emph{should} obey the functor laws, this is not
mandatory when declaring an instance of the \texthaskell{Functor} type
class. The first law can be stated as:
\begin{codehaskell}
fmap id = id
\end{codehaskell}
Polymorphism in Haskell allows us to write just \texthaskell{id} in
both sides of this law, but the types of each \texthaskell{id} are
different (which is more obvious when comparing this with
\eqref{eq:functor-identity}). A more precise way of stating the first
law in Haskell follows:
\begin{codehaskell}
fmap (id :: a -> a) = (id :: f a -> f a)
\end{codehaskell}
The second law can be stated as:
\begin{codehaskell}
fmap (g . f) = fmap g . fmap f
\end{codehaskell}
This law corresponds to \eqref{eq:functor-composition}. Proving both
of these laws amounts to proving that an instance of the
\texthaskell{Functor} type class is actually a functor.

\begin{remark}

  In the following examples, we shall prove the functor laws by hand.
  A different approach is that of \parencite{jeuring-et-al-2012},
  which develops a framework that supports testing such laws using
  QuickCheck.

\end{remark}

We have already described the identity endofunctor for any category.
In particular, there is the identity functor of \hask.

\begin{example}
  \label{ex:functor-identity-haskell}

  The identity functor of \hask, which is just an instance of the
  identity endofunctor as described in Example
  \ref{ex:functor-identity}, is defined as follows\footnote{Using the
    \texthaskell{InstanceSigs} language option.}:
  \begin{codehaskell}
newtype Identity a = Identity {unIdentity :: a}

instance Functor Identity where
  fmap :: (a -> b) -> Identity a -> Identity b
  fmap f (Identity x) = Identity (f x)
  \end{codehaskell}

  %% Note that \texthaskell{Identity a} is a renamed data type which is
  %% isomorphic to \texthaskell{a}. See \parencite[§
  %%   4.2.3]{peytonjones-2003}.

\end{example}

Common examples of functors in Haskell include \texthaskell{Maybe} and
lists. We shall use these functors again when studying natural
transformations and monads.

\begin{example}
  \label{ex:functor-maybe-haskell}

  %% \parencites[147--148]{lipovaca-2011}[246]{osullivan-et-al-2008}[20--21]{yorgey-2009}

  A basic example of functors in Haskell is the \texthaskell{Maybe}
  functor. Its type constructor is the \texthaskell{Maybe} type
  constructor:
  \begin{codehaskell}
data Maybe a = Nothing | Just a
  \end{codehaskell}
  Its \texthaskell{fmap} function is defined as follows:
  \begin{codehaskell}
instance Functor Maybe where
  fmap :: (a -> b) -> Maybe a -> Maybe b
  fmap _ Nothing  = Nothing
  fmap f (Just x) = Just (f x)
  \end{codehaskell}
  Basically, the \texthaskell{Maybe} type constructor sends types
  \texthaskell{a} to types \texthaskell{Maybe a}, a value of type
  \texthaskell{Maybe a} either contains a value of type
  \texthaskell{a} or is empty, and the \texthaskell{fmap} function
  sends functions \texthaskell{a -> b} to functions \texthaskell{Maybe
    a -> Maybe b}. This instance satisfies the laws. First, we prove
  that \eqref{eq:functor-identity} holds.

  \vspace{1em}
  \case{\texthaskell{Nothing}}
  \begin{steps}
    \steph{fmap id Nothing}
      \eqbydefh{fmap}
    \steph{Nothing}
      \eqbydefh{id}
    \steph{id Nothing}
  \end{steps}
  \case{\texthaskell{(Just x)}}
  \begin{steps}
    \steph{fmap id (Just x)}
      \eqbydefh{fmap}
    \steph{Just (id x)}
      \eqbydefh{id}
    \steph{Just x}
      \eqbydefh{id}
    \steph{id (Just x)}
  \end{steps}
  Second, we prove that \eqref{eq:functor-composition} holds.

  \vspace{1em}
  \case{\texthaskell{Nothing}}
  \begin{steps}
    \steph{(fmap g . fmap f) Nothing}
      \eqbydefh{(.)}
    \steph{fmap g (fmap f Nothing)}
      \eqbydefh{fmap}
    \steph{fmap g Nothing}
      \eqbydefh{fmap}
    \steph{Nothing}
      \eqbydefh{fmap}
    \steph{fmap (g . f) Nothing}
  \end{steps}
  \case{\texthaskell{Just x}}
  \begin{steps}
    \steph{fmap (g . f) (Just x)}
      \eqbydefh{fmap}
    \steph{Just ((g . f) x)}
      \eqbydefh{(.)}
    \steph{Just (g (f x))}
      \eqbydefh{fmap}
    \steph{fmap g (Just (f x))}
      \eqbydefh{fmap}
    \steph{fmap g (fmap f (Just x))}
      \eqbydefh{(.)}
    \steph{(fmap g . fmap f) (Just x)}
  \end{steps}

\end{example}

\begin{example}
  \label{ex:functor-list-haskell}

  %% \parencites[146--147]{lipovaca-2011}[20--21]{yorgey-2009}

  Another common example of functors in Haskell is the
  \texthaskell{[]} (list) functor. Its type constructor is
  \texthaskell{[]}, and its \texthaskell{fmap} function is the
  \texthaskell{map} function:
  \begin{codehaskell}
instance Functor [] where
  fmap :: (a -> b) -> [a] -> [b]
  fmap _ []     = []
  fmap f (x:xs) = f x : fmap f xs
  \end{codehaskell}
  Or:
  \begin{codehaskell}
instance Functor [] where
  fmap :: (a -> b) -> [a] -> [b]
  fmap = map
  \end{codehaskell}
  The \texthaskell{[]} type constructor sends types \texthaskell{a} to
  types \texthaskell{[a]}, that is, lists of \texthaskell{a}, and the
  \texthaskell{fmap} or \texthaskell{map} function, which we talked
  about earlier, sends functions of type \texthaskell{a -> b} to
  functions of type \texthaskell{[a] -> [b]}. In order to see that
  this instance satisfies the laws, we begin by proving
  \eqref{eq:functor-identity} by induction.

  \vspace{1em}
  \case{\texthaskell{[]}}
  \begin{steps}
    \steph{fmap id []}
      \eqbydefh{fmap}
    \steph{[]}
      \eqbydefh{id}
    \steph{id []}
  \end{steps}
  \case{\texthaskell{(x:xs)}}
  \begin{steps}
    \steph{fmap id (x:xs)}
      \eqbydefh{fmap}
    \steph{id x : fmap id xs}
      \eqbydefh{id}
    \steph{x : fmap id xs}
      \eqbyihh{}
    \steph{x:xs}
      \eqbydefh{id}
    \steph{id (x:xs)}
  \end{steps}
  Second, we prove \eqref{eq:functor-composition}, also by induction.

  \vspace{1em}
  \case{\texthaskell{[]}}
  \begin{steps}
    \steph{(fmap g . fmap f) []}
      \eqbydefh{(.)}
    \steph{fmap g (fmap f [])}
      \eqbydefh{fmap}
    \steph{fmap g []}
      \eqbydefh{fmap}
    \steph{[]}
      \eqbydefh{fmap}
    \steph{fmap (g . f) []}
  \end{steps}
  \case{\texthaskell{(x:xs)}}
  \begin{steps}
    \steph{fmap (g . f) (x:xs)}
      \eqbydefh{fmap}
    \steph{(g . f) x : fmap (g . f) xs}
      \eqbydefh{(.)}
    \steph{g (f x) : fmap (g . f) xs}
      \eqbyihh{}
    \steph{g (f x) : (fmap g . fmap f) xs}
      \eqbydefh{(.)}
    \steph{g (f x) : fmap g (fmap f xs)}
      \eqbydefh{fmap}
    \steph{fmap g (f x : fmap f xs)}
      \eqbydefh{fmap}
    \steph{fmap g (fmap f (x:xs))}
      \eqbydefh{(.)}
    \steph{(fmap g . fmap f) (x:xs)}
  \end{steps}

\end{example}

Additional examples of functors in Haskell include products and
coproducts, which are some of the constructions we discussed in the
previous chapter, and functions, which is an interesting case for
better understanding the type signature of \texthaskell{fmap}.

\begin{example}
  \label{ex:functor-product-haskell}

  %% \parencites[21]{yorgey-2009}

  Another usual example of functors in Haskell is the product functor.
  Its type constructor is \texthaskell{(,) a} (see Example
  \ref{ex:product-haskell}), and its \texthaskell{fmap} function is
  uniquely defined as follows:
  \begin{codehaskell}
instance Functor ((,) a) where
  fmap :: (b -> c) -> (a,b) -> (a,c)
  fmap f (x, y) = (x, f y)
  \end{codehaskell}
  Let us prove that this instance satisfies the functor laws. In the
  first place, we show that \eqref{eq:functor-identity} holds.
  \begin{steps}
    \steph{fmap id (x, y)}
      \eqbydefh{fmap}
    \steph{(x, id y)}
      \eqbydefh{id}
    \steph{(x, y)}
      \eqbydefh{id}
    \steph{id (x, y)}
  \end{steps}
  In the second place, we show that \eqref{eq:functor-composition}
  holds:
  \begin{steps}
    \steph{(fmap h . fmap g) (x, y)}
      \eqbydefh{(.)}
    \steph{fmap h (fmap g (x, y))}
      \eqbydefh{fmap}
    \steph{fmap h (x, g y)}
      \eqbydefh{fmap}
    \steph{(x, h (g y))}
      \eqbydefh{(.)}
    \steph{(x, (h . g) y)}
      \eqbydefh{fmap}
    \steph{fmap (h . g) (x, y)}
  \end{steps}

\end{example}

\begin{example}
  \label{ex:functor-coproduct-haskell}

  %% \parencites[149]{lipovaca-2011}[21]{yorgey-2009}

  In a similar way, the coproduct functor is an usual example of
  functors in Haskell. Its type constructor is \texthaskell{Either a}
  (see Example \ref{ex:functor-coproduct-haskell}), and its
  \texthaskell{fmap} function is uniquely defined as follows:
  \begin{codehaskell}
instance Functor (Either a) where
  fmap :: (b -> c) -> Either a b -> Either a c
  fmap _ (Left x)  = Left x
  fmap g (Right y) = Right (g y)
  \end{codehaskell}
  In order to see that this instance obeys the laws, we prove, in the
  first place, that \eqref{eq:functor-identity} holds.

  \vspace{1em}
  \case{\texthaskell{(Left x)}}
  \begin{steps}
    \steph{fmap id (Left x)}
      \eqbydefh{fmap}
    \steph{Left x}
      \eqbydefh{id}
    \steph{id (Left x)}
  \end{steps}
  \case{\texthaskell{(Right y)}}
  \begin{steps}
    \steph{fmap id (Right y)}
      \eqbydefh{fmap}
    \steph{Right (id y)}
      \eqbydefh{id}
    \steph{Right y}
      \eqbydefh{id}
    \steph{id (Right y)}
  \end{steps}
  In the second place, we prove that \eqref{eq:functor-composition}
  holds.

  \vspace{1em}
  \case{\texthaskell{(Left x)}}
  \begin{steps}
    \steph{(fmap h . fmap g) (Left x)}
      \eqbydefh{(.)}
    \steph{fmap h (fmap g (Left x))}
      \eqbydefh{fmap}
    \steph{fmap h (Left x)}
      \eqbydefh{fmap}
    \steph{Left x}
      \eqbydefh{fmap}
    \steph{fmap (h . g) (Left x)}
  \end{steps}
  \case{\texthaskell{(Right y)}}
  \begin{steps}
    \steph{(fmap h . fmap g) (Right y)}
      \eqbydefh{(.)}
    \steph{fmap h (fmap g (Right y))}
      \eqbydefh{fmap}
    \steph{fmap h (Right (g y))}
      \eqbydefh{fmap}
    \steph{Right (h (g y))}
      \eqbydefh{(.)}
    \steph{Right ((h . g) y)}
      \eqbydefh{fmap}
    \steph{fmap (h . g) (Right y)}
  \end{steps}

\end{example}

\begin{example}
  \label{ex:functor-function-haskell}

  %% \parencites[220--222]{lipovaca-2011}[21]{yorgey-2009}

  An interesting example of functors in Haskell is the function
  functor. Its type constructor is \texthaskell{(->) a}, and its
  \texthaskell{fmap} function is function composition:
  \begin{codehaskell}
instance Functor ((->) a) where
  fmap :: (b -> c) -> (a -> b) -> a -> c
  fmap g f = \x -> g (f x)
  \end{codehaskell}
  Or, equivalently:
  \begin{codehaskell}
instance Functor ((->) a) where
  fmap :: (b -> c) -> (a -> b) -> a -> c
  fmap = (.)
  \end{codehaskell}
  Let us see that this instance satisfies the functor laws. In the
  first place, we prove that \eqref{eq:functor-identity} holds.
  \begin{steps}
    \steph{fmap id f}
      \eqbydefh{fmap}
    \steph{id . f}
      \eqby{\eqref{eq:category-identity}}
    \steph{f}
      \eqbydefh{id}
    \steph{id f}
  \end{steps}
  In the second place, we prove that \eqref{eq:functor-composition}
  holds.
  \begin{steps}
    \steph{(fmap h . fmap g) f}
      \eqbydefh{(.)}
    \steph{fmap h (fmap g f)}
      \eqbydefh{fmap}
    \steph{fmap h (g . f)}
      \eqbydefh{fmap}
    \steph{h . (g . f)}
      \eqby{\eqref{eq:category-associativity}}
    \steph{(h . g) . f}
      \eqbydefh{fmap}
    \steph{fmap (h . g) f}
  \end{steps}

\end{example}

Since the functor laws are not part of the definition of the
\texthaskell{Functor} type class, we can declare instances which do
not satisfy the laws. In the following examples, we consider incorrect
definitions of the \texthaskell{Maybe} and \texthaskell{[]} functors.

\begin{example}
  \label{ex:functor-bad-maybe-haskell}

  It is possible to incorrectly define the \texthaskell{Maybe} functor
  (see Example \ref{ex:functor-maybe-haskell}) as follows:
  \begin{codehaskell}
instance Functor Maybe where
  fmap :: (a -> b) -> Maybe a -> Maybe b
  fmap _ Nothing  = Nothing
  fmap _ (Just _) = Nothing
  \end{codehaskell}
  Even though this definition of the \texthaskell{fmap} function is
  accepted by the Haskell type checker, the following counterexample
  proves that it violates the first functor law:
  \begin{codehaskell}
> fmap id (Just 0)
Nothing
> id (Just 0)
Just 0
  \end{codehaskell}
  Note that this is the only way to incorrectly declare the
  \texthaskell{Maybe} functor within \hask.

\end{example}

\begin{example}
  \label{ex:functor-bad-list-haskell}

  We can wrongly define the \texthaskell{[]} (list) functor too (see
  Example \ref{ex:functor-list-haskell}). Here is the declaration we
  discussed at the beginning of the chapter:
  \begin{codehaskell}
instance Functor [] where
  fmap :: (a -> b) -> [a] -> [b]
  fmap _ []     = []
  fmap f (x:xs) = f x : f x : fmap f xs
  \end{codehaskell}
  But this is not the only way. For instance:
  \begin{codehaskell}
instance Functor [] where
  fmap :: (a -> b) -> [a] -> [b]
  fmap _ []     = []
  fmap f (x:xs) = [f x]
  \end{codehaskell}
  However, neither of these instances satisfy the first functor law,
  as demonstrated by the following counterexamples. In the first case:
  \begin{codehaskell}
> fmap id [0,1]
[0,0,1,1]
> id [0,1]
[0,1]
  \end{codehaskell}
  In the second case:
  \begin{codehaskell}
> fmap id [0,1]
[0]
> id [0,1]
[0,1]
  \end{codehaskell}

\end{example}

\section{Functors in Agda}
\label{sec:functors-agda}

The Agda standard library defines functors in Agda just like functors
in Haskell, that is, without the functor laws\footnote{See
  \parencite[module \module{Category.Functor}]{danielsson-2013}.}.
Abel defines functors in the module \module{Abel.Category.Functor},
which includes the functor laws\footnote{We refer to propositional
  (intensional) equality, as defined in \parencite[module
    \module{Relation.Binary.PropositionalEquality}]{danielsson-2013}.}:
\begin{codeagda}
record Functor (F : Set → Set) : Set₁ where

  constructor mkFunctor

  field

    fmap    : {A B : Set} → (A → B) → F A → F B

    fmap-id : {A : Set} (fx : F A) → fmap id fx ≡ id fx

    fmap-∘  : {A B C : Set} {f : A → B} {g : B → C}
              (fx : F A) → fmap (g ∘ f) fx ≡ (fmap g ∘ fmap f) fx
\end{codeagda}
The inclusion of the functor laws makes it impossible to define a
functor which is not really a functor because all instances must prove
that \textagda{F} and \textagda{fmap} satisfy the laws.

As examples, we consider all instances described in Haskell in the
previous section.

\begin{example}[See module \module{Abel.Data.Maybe.Functor}]
  \label{ex:functor-maybe-agda}

  The \textagda{Maybe} functor in Agda is defined as follows:
  \begin{codeagda}
functor : Functor Maybe
functor = mkFunctor fmap fmap-id fmap-∘
  where
    fmap : {A B : Set} → (A → B) → Maybe A → Maybe B
    fmap f (just x) = just (f x)
    fmap _ nothing  = nothing

    fmap-id : {A : Set} (mx : Maybe A) → fmap id mx ≡ id mx
    fmap-id (just _) = refl
    fmap-id nothing  = refl

    fmap-∘ : {A B C : Set} {f : A → B} {g : B → C}
             (mx : Maybe A) → fmap (g ∘ f) mx ≡ (fmap g ∘ fmap f) mx
    fmap-∘ (just _) = refl
    fmap-∘ nothing  = refl
  \end{codeagda}
  This functor corresponds to the \texthaskell{Maybe} functor in
  Haskell (see Example \ref{ex:functor-maybe-haskell}).

\end{example}

\begin{example}[See module \module{Abel.Data.List.Functor}]
  \label{ex:functor-list-agda}

  The following instance corresponds to the \textagda{List} functor in
  Agda:
  \begin{codeagda}
functor : Functor List
functor = mkFunctor fmap fmap-id fmap-∘
  where
    fmap : {A B : Set} → (A → B) → List A → List B
    fmap _ []       = []
    fmap f (x ∷ xs) = f x ∷ fmap f xs

    fmap-id : {A : Set} (xs : List A) → fmap id xs ≡ id xs
    fmap-id []       = refl
    fmap-id (x ∷ xs) = cong (_∷_ x) (fmap-id xs)

    fmap-∘ : {A B C : Set} {f : A → B} {g : B → C}
             (xs : List A) → fmap (g ∘ f) xs ≡ (fmap g ∘ fmap f) xs
    fmap-∘             []       = refl
    fmap-∘ {f = f} {g} (x ∷ xs) = cong (_∷_ (g (f x))) (fmap-∘ xs)
  \end{codeagda}
  This definition matches that of the \texthaskell{[]} functor in
  Haskell (see Example \ref{ex:functor-list-haskell}).

\end{example}

\begin{example}[See module \module{Abel.Data.Product.Functor}]
  \label{ex:functor-product-agda}

  Here is the declaration of the product functor in Agda (see Example
  \ref{ex:product-agda}):
  \begin{codeagda}
functor : {A : Set} → Functor (_×_ A)
functor {A} = mkFunctor fmap fmap-id fmap-∘
  where
    fmap : {B C : Set} → (B → C) → A × B → A × C
    fmap g (x , y) = x , g y

    fmap-id : {B : Set} (x,y : A × B) → fmap id x,y ≡ id x,y
    fmap-id (x , y) = refl

    fmap-∘ : {B C D : Set} {g : B → C} {h : C → D}
             (x,y : A × B) → fmap (h ∘ g) x,y ≡ (fmap h ∘ fmap g) x,y
    fmap-∘ (x , y) = refl
  \end{codeagda}
  Compare this with the product functor in Haskell (see Example
  \ref{ex:functor-product-haskell}).

\end{example}

\begin{example}[See module \module{Abel.Data.Sum.Functor}]
  \label{ex:functor-coproduct-agda}

  The coproduct functor in Agda (see Example \ref{ex:coproduct-agda})
  is defined as follows:
  \begin{codeagda}
functor : {A : Set} → Functor (_+_ A)
functor {A} = mkFunctor fmap fmap-id fmap-∘
  where
    fmap : {B C : Set} → (B → C) → A + B → A + C
    fmap _ (inj₁ x) = inj₁ x
    fmap g (inj₂ y) = inj₂ (g y)

    fmap-id : {B : Set} (x+y : A + B) → fmap id x+y ≡ id x+y
    fmap-id (inj₁ _) = refl
    fmap-id (inj₂ _) = refl

    fmap-∘ : {B C D : Set} {g : B → C} {h : C → D}
             (x+y : A + B) → fmap (h ∘ g) x+y ≡ (fmap h ∘ fmap g) x+y
    fmap-∘ (inj₁ _) = refl
    fmap-∘ (inj₂ _) = refl
  \end{codeagda}
  Compare this with the coproduct functor in Haskell (see Example
  \ref{ex:functor-coproduct-haskell}).

\end{example}

\begin{example}[See module \module{Abel.Function.Functor}]
  \label{ex:functor-function-agda}

  Here is the definition of the function functor in Agda, which
  corresponds to the Haskell functor described in Example
  \ref{ex:functor-function-haskell}:
  \begin{codeagda}
functor : {A : Set} → Functor (λ B → A → B)
functor = mkFunctor (λ g f → g ∘ f) (λ _ → refl) (λ _ → refl)
  \end{codeagda}

\end{example}

\begin{example}
  \label{ex:functor-bad-maybe-agda}

  We can try to define the alternative \texthaskell{Maybe} functor of
  Example \ref{ex:functor-bad-maybe-haskell} in Agda:
  \begin{codeagda}
functor : Functor Maybe
functor = mkFunctor fmap fmap-id fmap-∘
  where
    fmap : {A B : Set} → (A → B) → Maybe A → Maybe B
    fmap f (just x) = nothing
    fmap _ nothing  = nothing

    fmap-id : {A : Set} (mx : Maybe A) → fmap id mx ≡ id mx
    fmap-id (just _) = ?
    fmap-id nothing  = refl

    fmap-∘ : {A B C : Set} {f : A → B} {g : B → C}
             (mx : Maybe A) → fmap (g ∘ f) mx ≡ (fmap g ∘ fmap f) mx
    fmap-∘ (just _) = refl
    fmap-∘ nothing  = refl
  \end{codeagda}
  But this code does not type check because there is a proof missing.
  As we saw in Example \ref{ex:functor-bad-maybe-haskell}, the first
  functor law does not hold for this definition of the
  \textagda{Maybe} functor, so there is no way to make this instance
  type check in Agda.

\end{example}

\section{References}
\label{sec:functors-references}

The definition of a functor is based on
\parencites[13]{maclane-1998}[428]{poigne-1992}, the power set and
identity functors are taken from \parencites[431]{poigne-1992} and
\parencite[11]{marquis-2013}, respectively, and the study of functors
in Haskell is based on \parencites[146--150,
  218--227]{lipovaca-2011}[18--23]{yorgey-2009}.

\clearemptydoublepage
