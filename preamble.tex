\usepackage{polyglossia}
\setdefaultlanguage{english}
\setmainlanguage[variant=usmax]{english}

%% \usepackage[showframe]{geometry}

\hyphenation{i-so-mor-phism}
\hyphenation{Grund-leh-ren}

\usepackage{amsmath}

\usepackage{enumitem}
\usepackage{epigraph}

\setlength{\epigraphrule}{0pt}
%% \setlength{\epigraphwidth}{0.4\textwidth}
\setlength{\epigraphwidth}{0.55\textwidth}

\usepackage{fancyhdr}
\pagestyle{fancy}

\renewcommand{\chaptermark}[1]{
  \ifnum\value{chapter}>0
    \markboth{\thechapter.\ #1}{}
  \else
    \markboth{#1}{}
  \fi
}

\fancyhead{}
\fancyhead[ER]{\textsl{\nouppercase{\leftmark}}}
\fancyhead[OL]{\textsl{\nouppercase{\rightmark}}}

\fancyhead[LE,RO]{\thepage}

\fancyfoot{}

\renewcommand{\headrulewidth}{0pt}
\renewcommand{\footrulewidth}{0pt}

\newcommand{\clearemptydoublepage}{\clearpage{\pagestyle{empty}\cleardoublepage}}

\usepackage{fontspec}

\usepackage{subcaption}

\usepackage{unicode-math}

\setmathfont{latinmodern-math.otf}

%% \setmathfont[range={\llparenthesis,\rrparenthesis}]{Asana-Math.otf}
\setmathfont[range={\llparenthesis,\rrparenthesis}]{xits-math.otf}

%% \usepackage[only,llparenthesis,rrparenthesis]{stmaryrd}

\setmonofont
  [ BoldFont       = DejaVuSansMono-Bold.ttf,
    ItalicFont     = DejaVuSansMono-Oblique.ttf,
    BoldItalicFont = DejaVuSansMono-BoldOblique.ttf,
    Scale          = MatchLowercase,
  ]
  {DejaVuSansMono.ttf}

\usepackage{xcolor}

\usepackage{xltxtra}
\usepackage{xunicode}

\usepackage{makeidx}

\usepackage{xspace}

\usepackage{amsthm}

\let\equation\gather
\let\endequation\endgather

\usepackage{tikz}
\tikzset{node distance=2.5cm,auto}

%%%%%%%%%%%%%%%%%%%%%%%%%%%%%%%%%%%%%%%%%%%%%%%%%%%%%%%%%%%%%%%%%%%%%%%%%%%%%%
%% biblatex

\usepackage[style=authoryear,backref=true]{biblatex}
\usepackage[english=american]{csquotes}

\SetCiteCommand{\parencite}

\DeclareFieldFormat[article]{title}{#1}
\DeclareFieldFormat[inbook]{title}{#1}
\DeclareFieldFormat[inproceedings]{title}{#1}
\DeclareFieldFormat[thesis]{title}{#1}

\renewcommand{\subtitlepunct}{\ifterm{}{\addcolon}\addspace}

\addbibresource{cain.bib}
\addbibresource{/home/jpvi/etc/bible/bible.bib}

\DefineHyphenationExceptions{english}{
  type-class-o-pe-dia Grund-leh-ren
}

%%%%%%%%%%%%%%%%%%%%%%%%%%%%%%%%%%%%%%%%%%%%%%%%%%%%%%%%%%%%%%%%%%%%%%%%%%%%%%
%% hyperref

\usepackage{hyperref}

\hypersetup{colorlinks=true}
\hypersetup{linktocpage=true}

\definecolor{citecolor}{rgb}{0,0.7,0}
\definecolor{linkcolor}{rgb}{0.7,0,0}
\definecolor{urlcolor}{rgb}{0,0,0.7}

\hypersetup{citecolor=citecolor}
\hypersetup{linkcolor=linkcolor}
\hypersetup{urlcolor=urlcolor}

\ifcolor
\else
  \hypersetup{hidelinks}
\fi

%% \makeindex %% detex -w cain.tex | sort -uf > foo

\hypersetup{pdftitle={Category Theory Applied to Functional Programming}}
\hypersetup{pdfauthor={Juan Pedro Villa Isaza}}
\hypersetup{pdfkeywords={category theory, functional programming}}
\hypersetup{pdfsubject={Systems Engineering undergraduate project}}

%%%%%%%%%%%%%%%%%%%%%%%%%%%%%%%%%%%%%%%%%%%%%%%%%%%%%%%%%%%%%%%%%%%%%%%%%%%%%%
%% todo
%%%%%%%%%%%%%%%%%%%%%%%%%%%%%%%%%%%%%%%%%%%%%%%%%%%%%%%%%%%%%%%%%%%%%%%%%%%%%%

\newif\iftodo

\todotrue

\iftodo
  \ifcolor
    \newcommand{\todo}[1]{\textcolor{red}{TODO}~(#1)}
  \else
    \newcommand{\todo}[1]{TODO~(#1)}
  \fi
\else
  \newcommand{\todo}[1]{}
\fi

%%%%%%%%%%%%%%%%%%%%%%%%%%%%%%%%%%%%%%%%%%%%%%%%%%%%%%%%%%%%%%%%%%%%%%%%%%%%%%
%% Verbatim
%%%%%%%%%%%%%%%%%%%%%%%%%%%%%%%%%%%%%%%%%%%%%%%%%%%%%%%%%%%%%%%%%%%%%%%%%%%%%%

\newfontfamily\dejavu{DejaVu Sans Mono}

\usepackage{fancyvrb}
\usepackage{relsize}

\DefineVerbatimEnvironment
  {codeagda}
  {Verbatim}
  {frame=lines}

\DefineVerbatimEnvironment
  {codehaskell}
  {Verbatim}
  {frame=lines}

\DefineVerbatimEnvironment
  {code}
  {Verbatim}
  {frame=lines}

\newcommand{\textagda}[1]{\texttt{#1}}
\newcommand{\texthaskell}[1]{\texttt{#1}}

\newcommand{\module}[1]{\textsf{#1}}

%%%%%%%%%%%%%%%%%%%%%%%%%%%%%%%%%%%%%%%%%%%%%%%%%%%%%%%%%%%%%%%%%%%%%%%%%%%%%%
%% Steps
%%%%%%%%%%%%%%%%%%%%%%%%%%%%%%%%%%%%%%%%%%%%%%%%%%%%%%%%%%%%%%%%%%%%%%%%%%%%%%

\newlist{steps}{enumerate}{1}
\setlist[steps]{label=\arabic*.,ref=\arabic*,itemsep=0pt}

\newcommand{\step}[1]{\item[]{#1}}
\newcommand{\steph}[1]{\step{\texthaskell{#1}}}
\newcommand{\stepm}[1]{\step{\ensuremath{#1}}}

\newcommand{\by}[1]{(by {#1})}

\newcommand{\bydef}[1]{(by definition of {#1})}
\newcommand{\bydefh}[1]{\bydef{\texthaskell{#1}}}

\newcommand{\byih}[1]{(by inductive hypothesis)}

\newcommand{\eq}{\ensuremath{=}}
\newcommand{\eqh}{\texthaskell{=}}

\newcommand{\eqby}[1]{\item[]\quad\eq\quad\by{#1}}
\newcommand{\eqbyh}[1]{\item[]\quad\eqh\quad\by{#1}}

\newcommand{\eqbydef}[1]{\item[]\quad\eq\quad\bydef{#1}}
\newcommand{\eqbydefh}[1]{\item[]\quad\eqh\quad\bydefh{#1}}

\newcommand{\eqbyih}{\item[]\quad\eq\quad\byih{}}
\newcommand{\eqbyihh}[1]{\item[]\quad\eqh\quad\byih{\texthaskell{#1}}}

\newcommand{\eqtext}[1]{\item[]\quad\eq\quad({#1})}

\newcommand{\eqbynothing}{\item[]\quad\eq\quad}

%%%%%%%%%%%%%%%%%%%%%%%%%%%%%%%%%%%%%%%%%%%%%%%%%%%%%%%%%%%%%%%%%%%%%%%%%%%%%%
%% Theorems
%%%%%%%%%%%%%%%%%%%%%%%%%%%%%%%%%%%%%%%%%%%%%%%%%%%%%%%%%%%%%%%%%%%%%%%%%%%%%%

\theoremstyle{definition}

\newtheorem{definition}{Definition}[chapter]

\newtheorem{example}{Example}[section]
\newtheorem{examples}[example]{Examples}

\newtheorem*{example*}{Example}

\theoremstyle{plain}

\newtheorem{theorem}{Theorem}[chapter]
\newtheorem{lemma}[theorem]{Lemma}
\newtheorem{proposition}[theorem]{Lemma}

\theoremstyle{remark}

\newtheorem{notation}{Notation}[chapter]
\newtheorem{note}{Note}
\newtheorem{remark}{Remark}[chapter]

\newtheorem*{remark*}{Remark}
\newtheorem*{terminology}{Terminology}

%%%%%%%%%%%%%%%%%%%%%%%%%%%%%%%%%%%%%%%%%%%%%%%%%%%%%%%%%%%%%%%%%%%%%%%%%%%%%%
%% Commands and operators
%%%%%%%%%%%%%%%%%%%%%%%%%%%%%%%%%%%%%%%%%%%%%%%%%%%%%%%%%%%%%%%%%%%%%%%%%%%%%%

%% Categories

\DeclareMathOperator{\obj}{O}
\DeclareMathOperator{\mor}{M}

\DeclareMathOperator{\dom}{dom}
\DeclareMathOperator{\cod}{cod}

\DeclareMathOperator{\id}{id}

\newcommand{\idO}[1]{\natO{\id}{#1}}

\newcommand{\comp}{\ensuremath{\mathrel{\circ}}}
\newcommand{\pmoc}{\ensuremath{\mathrel{;}}}

\newcommand{\cat}[1]{\ensuremath{\mathcal{#1}}}

\newcommand{\catO}[1]{\ensuremath{\cat{#1}_{\obj}}}
\newcommand{\catM}[1]{\ensuremath{\cat{#1}_{\mor}}}

\newcommand{\catbf}[1]{\ensuremath{\mathbf{#1}}\xspace}

\newcommand{\alg}{\catbf{Alg}}
\newcommand{\falg}[1]{\func{#1}\text{-}\alg}
\newcommand{\hask}{\catbf{Hask}}
\newcommand{\set}{\catbf{Set}}

%% Functors

\newcommand{\func}[1]{\ensuremath{\mathsf{#1}}}

\newcommand{\funcO}[1]{\ensuremath{\func{#1}_{\obj}}}
\newcommand{\funcM}[1]{\ensuremath{\func{#1}_{\mor}}}

%% Natural transformations

\newcommand{\nat}[1]{\ensuremath{#1}}

\newcommand{\natO}[2]{\ensuremath{\nat{#1}_{#2}}}

%% Monads

\newcommand{\underscore}{\ensuremath{\underline{~}}}

\newcommand{\mon}[1]{\func{#1}}

\newcommand{\monM}[2]{\ensuremath{#2^*}}

%% Algebras

\DeclareMathOperator{\inmo}{in}

\newcommand{\cata}[1]{\ensuremath{\llparenthesis#1\rrparenthesis}}

\DeclareMathOperator{\zero}{zero}
\DeclareMathOperator{\suc}{suc}

\newcommand{\listof}[1]{\text{List}(#1)}

\DeclareMathOperator{\nil}{nil}
\DeclareMathOperator{\cons}{cons}

\DeclareMathOperator{\fold}{fold}
\DeclareMathOperator{\foldr}{foldr}

\DeclareMathOperator{\add}{add}
\DeclareMathOperator{\mul}{mul}

\DeclareMathOperator{\length}{length}
\DeclareMathOperator{\append}{append}
\DeclareMathOperator{\map}{map}



%% \newcommand{\term}[1]{#1}
